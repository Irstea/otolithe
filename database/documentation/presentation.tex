\chapter{Le logiciel Otolithe}
\section{Présentation}

Le logiciel Otolithe a été conçu pour faciliter la lecture des pièces calcifiées de poissons (otolithes, écailles, rayons de nageoires, etc.). Il permet à chaque lecteur de positionner des points sur des photos, et de comparer ensuite chaque lecture.

Il a été conçu pour l'unité de recherche \textit{Écosystèmes aquatiques et changements globaux} d'IRSTEA, à Cestas (33).

La première version est parue à l'automne 2013, la version 2.0 est sortie en août 2018.

Le code comprend environ 6400 lignes (commentaires compris), dont 2200 concernent l'affichage des pages web. Il a été écrit en PHP, les pages web sont générées en HTML et Javascript avec le composant Smarty.

\section{Fonctionnalités générales}

L'application est organisée autour de la notion d'expérimentation : une expérimentation regroupe les poissons qui font l'objet d'une lecture par un groupe de lecteurs définis.

Un poisson peut être rattaché à plusieurs expérimentations.

Les poissons peuvent être saisis ou importés, à partir d'un fichier au format CSV.

Pour un poisson, on peut décrire une ou plusieurs pièces calcifiées, et rattacher à celles-ci une ou plusieurs photos.

Les lecteurs positionnent des points sur les photos. Une fois toutes les lectures réalisées, il est possible de les afficher simultanément et, au besoin, réaliser une lecture consensuelle.

Les résultats des lectures peuvent être exportées dans un fichier au format CSV.

Les lecteurs ne peuvent accéder qu'aux poissons qui font partie des expérimentations qui leur sont ouvertes : il est ainsi possible de gérer des lots de poissons différents avec des lecteurs différents, tout en conservant la confidentialité des lectures.

\section{Technologie employée}
\subsection{Base de données}

L'application a été conçue pour fonctionner avec Postgresql, en version 9.5. 

\subsection{Langage de développement et framework utilisé}
Le logiciel a été écrit en PHP, en s'appuyant sur le framework \textit{Prototypephp} \cite{prototypephp}, développé parallèlement par l'auteur du logiciel.

Il utilise la classe \textit{Smarty} \cite{smarty} pour gérer l'affichage des pages HTML. Celles-ci sont générées en utilisant \textit{Jquery} \cite{jquery}  et divers composants associés. Le rendu général est réalisé avec \textit{Bootstrap} \cite{bootstrap}.



\subsection{Liste des composants externes utilisés}
% \usepackage{array} is required
\begin{longtable}{|>{\raggedright\arraybackslash}p{3cm}|c|c|>{\raggedright\arraybackslash}p{3cm}|>{\raggedright\arraybackslash}p{3cm}|}
\hline 
\textbf{Nom} & \textbf{Version} & \textbf{Licence} & \textbf{Usage} & \textbf{Site} \\ 
\hline 
\endhead
PrototypePHP & 4/7/2018 & LGPL & Framework & \href{https://github.com/equinton/prototypephp}{github.com/ equinton/ prototypephp} \\ 
\hline 
Smarty & 3.1.31 & LGPL & Générateur de pages HTML & \href{http://www.smarty.net}{www.smarty.net} \\ 
\hline 
Smarty-gettext & 1.1.1 & LGPL & Gestion du multi-linguisme avec Smarty & \href{https://github.com/smarty-gettext/smarty-gettext}{https://github.com/smarty-gettext/smarty-gettext} \\
\hline
PHPCAS & 1.3.5 & Apache 2.0 & Identification auprès d'un serveur CAS & \href{https://wiki.jasig.org/display/CASC/phpCAS}{wiki.jasig.org/ display/ CASC/ phpCAS} \\ 
\hline 
Bootstrap & 3.3.7 & MIT & Présentation HTML & \href{http://getbootstrap.com}{get.bootstrap.com} \\ 
\hline 
js-cookie-master & 2.1.4 & MIT & Gestion des cookies dans le navigateur & \href{https://github.com/js-cookie/js-cookie}{github.com/ js-cookie/ js-cookie} \\ 
\hline 
Datatables & 1.10.15 & MIT & Affichage des tableaux HTML & \href{http://www.datatables.net/}{www.datatables. net} \\ 
\hline 
Datetime-moment &  & MIT & Formatage des dates dans les tableaux & \href{https://datatables.net/plug-ins/sorting/datetime-moment}{datatables.net/ plug-ins/ sorting/ datetime-moment} \\ 
\hline 
Moment &  & MIT & Composant utilisé par datetime-moment & \href{http://momentjs.com} {momentjs.com}\\ 
\hline 
JQuery & 3.3.1 & $\approx$ BSD & Commandes Javascript & \href{http://jquery.com/}{jquery.com} \\ 
\hline 
JQuery-ui & 1.12.1 & $\approx$ BSD & Commandes Javascript pour les rendus graphiques & \href{http://jqueryui.com/}{jqueryui.com} \\ 
\hline 
Jquery-timepicker-addon &  & MIT & Time picker & \href{https://github.com/trentrichardson/jQuery-Timepicker-Addon}{github.com/ trentrichardson/ jQuery-Timepicker-Addon} \\ 
\hline 
Magnific-popup & 1.1.0 & MIT & Affichage des photos & \href{http://dimsemenov.com/plugins/magnific-popup/}{dimsemenov .com/plugins/ magnific-popup/}\\ 
\hline 
Smartmenus &  & MIT & Génération du menu HTML & \href{http://www.smartmenus .org}{www.smartmenus .org} \\ 
\hline 
AlpacaJS & 1.5.23 & Apache 2 & Génération et saisie des métadonnées (pour une version future) & 
\href{http://www.alpacajs.org/}{www.alpacajs.org}\\
\hline
\caption{Table des composants externes utilisés dans l'application}
\end{longtable} 

\section{Sécurité}

L'application a été conçue pour résister aux attaques dites opportunistes selon la nomenclature ASVS v3 \cite{asvs} de l'OWASP \cite{owasp}. Des tests d'attaque ont été réalisés en août 2018 avec le logiciel ZapProxy \cite{zaproxy}, et n'ont pas détecté de faiblesse particulière.

La gestion des droits est conçue pour qu'un lecteur ne puisse accéder qu'aux poissons faisant partie de ou des expérimentations auxquelles il est rattaché. 

Hormis le droit \textit{admin} qui permet de tout faire dans le logiciel, y compris paramétrer la gestion des droits, les droits suivants sont gérés :
\begin{itemize}
\item gestion : permet de modifier les données concernant un poisson, ajouter une photo, etc.
\item gestionCompte : permet de rajouter des comptes, de créer une expérimentation et d'y rattacher des lecteurs.
\end{itemize}

Les lecteurs ne peuvent que réaliser des lectures ou modifier les leurs.

\section{Licence}
Le logiciel est diffusé selon les termes de la licence GNU AFFERO GENERAL PUBLIC LICENSE version 3, en date du 19 novembre 2007 \cite{agpl}.

\section{Copyright}

L'application est en cours de dépôt auprès de l'Agence de protection des programmes \cite{app}.
